\documentclass{tufte-handout}

\usepackage{lmodern}
\usepackage{amsmath}
\usepackage{amssymb}

\newcounter{example}
\newenvironment{example}
{\refstepcounter{example}\begin{quote}
\textbf{Example \arabic{example}}}
{

$\square$\end{quote}}

\newenvironment{exercise}
{\refstepcounter{example}\begin{quote}
\textbf{Exercise \arabic{example}}}
{\end{quote}}

\newcommand{\RED}[1]{\textcolor{red}{#1}}
\newcommand{\TODO}[1]{\RED{TODO: #1}}

\title[So You Think You Can Prove?]{So You Think You Can Prove?\\
{\large A Practical Tutorial in Writing Formal Proofs}}

\author{Elias Castegren, Ellen Arvidsson}

\synctex=1

\begin{document}
\maketitle

\marginnote{This document was developed for the course Semantics
  of Programming Languages at Uppsala University. Please send any
  feedback to \texttt{elias.castegren@it.uu.se}. }

\section{Motivation}

The purpose of this document is to teach you about mathematical
reasoning; in particular how to write \emph{formal proofs}. A
formal proof is a sequence of steps that transforms one
configuration of symbols into another configuration of symbols
(i.e. the thing you want to prove)\footnote{For example, the
  starting configuration might be $1 + (2 + 3)$, and the steps
  might be $1 + (2 + 3) = 1 + 5 = 6$}. Writing formal proofs isn't
always necessary to convince anyone of your point, but in order to
learn how to conduct mathematical reasoning it is important to
know the underlying formal techniques.

This document should not be seen as a beginner's text book in
logic or even formal proofs -- the text assumes that you have a
basic understanding of both logic and proofs. The first goal of
this document is to have you brush up on this understanding and
remind you about concepts and terminology. A large part of formal
proofs is quite mechanical, but learning how to operate the
machinery requires practice. The second goal is therefore to
provide you with a set of practical tools that you can use when
doing your own formal proofs.

The rest of this tutorial has three sections.
%
The first section covers how to formalise informal statements,
that is, how to translate natural language into the symbols that
your proofs will be dealing with.
%
The following section covers techniques for formal proofs, sorted
by the kinds of formal constructs you might come across and how to
handle them.
%
Finally, the last section motivates the concept of structural
induction. This section should be seen as bonus content for the
interested reader.
%
Each section comes with a couple of exercises. Suggested solutions
are provided at the end of this document.

\section{Formalising Mathematical Statements}

% - Formalisera matematiska påståenden
%   - Uppställning
%   - Exempel

Before you can even start to prove anything formally, you need to
know how to \emph{state} what it is you want to prove in a formal
way. Since formal proofs is all about careful manipulation of
symbols, you first need to be able to write the initial symbols
down so that you can start manipulating them. Sometimes an
exercise or assignment will give you a formal statement to begin
with, but most times you will not be so lucky.

When dealing with proofs, we typically call a statement that can
be true or false (but not both) a \emph{proposition}\footnote{In
  general we will use ``derivable'' instead of ``true'', since we
  might be looking at what we can derive with a given set of rules
  rather than be concerned with the concept of truth. }. Here are
three examples of propositions:
\begin{itemize}
\item ``Three is smaller than two''
\item ``There is at least one natural number between five
and ten''
\item ``Bananas are yellow''
\end{itemize}

\noindent
Note that even though the first example is clearly not true, it is
still a proposition!
%
In order to write the third statement down formally, we need a way
to write down the concepts of ``something being a banana'' and
``something beng yellow''. These are examples of
\emph{predicates}. Predicates can be thought of as functions whose
result is a proposition; we can write $\mathit{banana}(x)$ or
$\mathit{yellow}(x)$ for the propositions that $x$ (whatever it
is) is a banana or yellow respectively. Note that
$\mathit{yellow}$ is a predicate, while $\mathit{yellow}(x)$ is a
proposition.

\marginnote{\textbf{Glossary}\\
\begin{itemize}
\item A \emph{proposition} is a statement that is either true or false.
\item A \emph{predicate} is a function whose result is a proposition.
\item A \emph{relation} is a predicate which takes two or more arguments.
\end{itemize}
}

With these predicates, we can write the propositions formally as
follows:

\begin{itemize}
\item $3 < 2$
\item $\exists n : \mathbb{N} .~ 5 < n ~\land~ n < 10$
\item $\forall b. \mathit{banana}(b) \implies \mathit{yellow}(b)$
\end{itemize}

\noindent
Note that have not formally \emph{defined} how to identify a
banana or that an object is yellow. That may or may not be
important for what you want to do with these predicates, but the
important step for now is that there is a \emph{symbolic
  representation} of these properties.

Note also that we used (for example) the symbol $<$ without
introducing it. This is because it already has a meaning that one
can be expected to understand. Indeed, the symbol $<$ is also a
predicate, but one that takes two arguments (note how $3 < 2$ is a
proposition). A predicate that takes two or more arguments is
often called a \emph{relation}.

\begin{example}\label{busdriver}
  There is a Swedish children's song (to the tune of ``Oh
  Tannenbaum'') called ``En busschauff\"{o}r'' whose lyrics let us
  know that ``A bus driver is a man with a jolly
  mood''\footnote{Swedish children learn the logical concept of
    the \emph{contrapositive} from an early age. The song
    continues to state that ``if he does not have a jolly mood,
    then he is no bus driver'':
    $\forall p. \lnot \mathit{jolly}(p) \implies \lnot
    \mathit{busDriver}(p)$}. In order to formalise this
  proposition, we start by defining three predicates:
  \begin{itemize}
  \item $\mathit{busDriver}~p$ -- the person $p$ is a bus driver
  \item $\mathit{male}~p$ -- the person $p$ identifies as male
  \item $\mathit{jolly}~p$ -- the person $p$ has a jolly mood
  \end{itemize}
  The proposition of the song can now be formalised as ``for any
  person $p$, if $p$ is a bus driver, then $p$ is male and $p$ has
  a jolly mood'':
  \[
    \forall p.
    \mathit{busDriver}~p \implies
    \mathit{male}~p ~\land~
    \mathit{jolly}~p
  \]
  Again, we can learn from this example is that it is perfectly
  possible to formalise propositions which are not true; most of
  us will have met bus drivers who do not identify as male or who
  we would not classify as jolly.
\end{example}

One of the points of translating an informal statement to a formal
one is to get rid of the ambiguity that is prevalent in natural
language. For example, when we say that ``Bananas are yellow'', do
we mean that \emph{all} bananas are yellow or that \emph{there
  are} bananas which are yellow\footnote{This distinction turns
  out to be important: only the latter interpretation is true
  since there are bananas which are red!}. When formalising the
proposition, we get rid of this ambiguity:

\begin{itemize}
\item ``All bananas are yellow'' -- $\forall b. \mathit{banana}(b) \implies \mathit{yellow}(b)$
\item ``Some bananas are yellow'' -- $\exists b. \mathit{banana}(b) \land \mathit{yellow}(b)$
\end{itemize}

\noindent
The ambiguity of natural languages also shows up in how we use
conjunctions like ``or'' when we speak. If you asking someone if
they would like tea or coffee, you are implicitly expecting them
to pick one of them, and not both\footnote{While it is logically
  correct to answer ``yes'' to the question ``Would you like tea
  or coffee'', it is also quite obnoxious. Don't be that person.
}. In logic there is a distinction between ``inclusive or'' and
``exclusive or'', each with their own (clearly defined) semantics.

We will end this section with an example that is more connected to
computer science than the colour of bananas and the mood of bus
drivers.

\begin{example}
  A list is either empty or consists of one element followed by
  the rest of the list. A list can be \emph{filtered} given some
  predicate $P$, such that elements that do not satisfy $P$ are
  discarded. One important property of filtering is that if an
  element for which $P$ holds is in a list, this element will also
  be in the list if it is filtered with $P$\footnote{This is a
    property we want, but it is not enough. Try to think about a
    definition of filtering that is wrong but which still
    satisfies this property. }.

  If we assume the existence of a $\mathit{filter}$ operation and
  a relation $\mathit{member}$ that holds if a given integer is in
  a given list, we can formalise the main property as a
  proposition:
  \[
    \forall x, l.~P(x) ~\land~ \mathit{member}(x, l)
    \implies
    \mathit{member}(x, \mathit{filter}(l, P))
  \]
  We will go ahead and formalise this with enough details so that
  we can actually prove that what we have said holds (although
  \RED{the proof will be given in a later section}).

  We start by defining the grammar of lists of integers:

  $\mathit{list} ::= \epsilon ~|~ n; \mathit{list}$

  \noindent
  We then continue to define the operation $\mathit{filter}$ which
  produces a filtered list given some predicate $P$:

  $\mathit{filter(l, P)} =
  \begin{cases}
    \epsilon & \text{if } l = \epsilon\\
    n; \mathit{filter}(l', P) & \text{if } l = n; l' \text{ and } P(n) \text{ holds}\\
    \mathit{filter}(l', P) & \text{if } l = n; l' \text{ and } P(n) \text{ does not hold}\\
  \end{cases}$

  \noindent
  Finally, we define the derivation rules for when an element is a
  member of a list; it is either the first element of the list, or
  it is a member of the rest of the list:

  \[
    \frac{}{~\mathit{member}(x, x; \mathit{list})~}
    \qquad
    \frac{~\mathit{member}(x, \mathit{list})~}{~\mathit{member}(x, y; \mathit{list})~}
  \]

\end{example}

\subsection{Exercises}

\begin{enumerate}
\item\label{freshclean} The Outkast song ``So Fresh, So Clean''
  from the 2000 album Stankonia starts with the statement ``Ain't
  nobody dope as me, I'm just so fresh, so clean''. Formalise this
  statement, assuming this can be generalised so that any person
  who is both fresh and clean is more dope than anyone
  else\footnote{You can ignore the fact that this introduces a
    contradiction if there are two people who are both fresh and
    clean}. Start by introducing any predicates you might need.
  % ∀p. fresh(p) ∧ clean(p) => ∀q. dopeness q < dopeness p
\item The offside rule in soccer states that a player is in
  ``offside position'' if they are
  \begin{enumerate}
  \item in the opponents' half of the pitch;
  \item closer to the opponents' goal than the ball is; and
  \item closer to the opponents' goal than the second-closest
    opponent (the opponent closest to the goal is typically the
    goal keeper)
  \end{enumerate}

  Define the proposition ``player $p$ is in offside position''.
  For simplicity you may assume that the pitch is a
  one-dimensional line and that both the ball and the players are
  distinct points on this line. Write your proposition as an
  equivalence, i.e. $\forall p. \mathit{offside}(p) \iff \dots$,
  where $\dots$ is your definition. You may want to introduce
  helper functions for getting different positions on the pitch or
  for getting sets of players\footnote{\textbf{Hint:} Set
    comprehensions may also come in handy. For example, the set of
    all even natural numbers can be written as
    $\{n ~|~ n \in \mathbb{N} \land n~mod~2 = 0\}$}.
  % ∀p. offside(p) <=>
  %     pos(p) < goal(p)/2 ∧
  %     pos(p) < ball ∧
  %     forall q. q ∈ {q | q ∈ opponents(p) ∧ pos(q) > min(positions(opponents(p)))} ==> pos(p) < pos(q)
\end{enumerate}

\section{Writing Formal Proofs}

In this section we cover \TODO{Continue here}

\subsection{Implications and Assumptions}

\subsection{Quantifiers}

\subsection{Proof by Induction}

\subsection{Proof by Contradiction}

\subsection{Exercises}

% - Koncept (varje koncept har exempel)
%   - Implikationer och antaganden
%   - Kvantifierare
%   - (Strukturell) induktion
%   - Motsägelsebevis
% - Övningar

\section{Bonus: Motivation for Why Structural Induction Works}

% - Överkurs om induktion

\section{Suggested Solutions}

The selected solutions are omitted before the seminar.

\end{document}